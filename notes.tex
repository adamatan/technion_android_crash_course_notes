%Format: Latex
\documentclass{article}
\setlength{\textwidth}{17cm}
\setlength{\textheight}{9in}
\setlength{\topmargin}{-1.5cm}
\setlength{\oddsidemargin}{0in}
\setlength{\evensidemargin}{0in}

\usepackage{listings}

\begin{document}
\title{Android Crash Course - Notes}
\author{Typed by Adam Matan, \texttt{adam@matan.name} \\
\texttt{http://matan.name}}
\date{Technion, Israel, Summer 2012}

\maketitle

\begin{abstract}
This document is a very poor replacement for the course slides; Instead, it focuses on the topics that were discussed in class.
\end{abstract}


\section{UI}
\subsection{Layouts}
A layout is an object that defines the way UI elements are ordered.

\subsubsection{Relative layout}

The \texttt{RelativeLayout} layout defines the position of each element in relation to other elements and the layout itself. For example:

An \texttt{Activity} is preferred over a \texttt{Layout} because it is reusable. Suppose that we create an activity for logging in into the Technion user page. 

\begin{itemize}
\item The \texttt{gravity} property
\end{itemize}

\subsection{Menus and Dialogs}

Each Android device has a menu button. Older devices have a physical one, while newer devices have a touch area. The menu button is important because the user expects to interact with menu items ( \texttt{exit},  \texttt{settings} etc.) Therefore, an application with a menu is easier to navigate and operate. Each application is associated with a default menu, which has a \texttt{settings} item that does nothing.

A menu consists of menu items and submenus. The submenus are limited to one hierarchy level (i.e., there are no sub-sub menus).






\subsection{Starting level} Many graduate students
arrive in the department with no background at all in numerical analysis, or computers.
For this reason, the first part of the course will cover \textit{all} of
AM261 at an accelerated pace. \textit{ALL students} are
required to pass an examination on the material taught in AM261:
(Introduction to Numerical Analysis).
Students who are confident that they can obtain 80\% on an AM261 level exam can write such an exam
early, and skip the lectures devoted to that material.

\subsection{Textbooks}
\begin{itemize}
\item
The AM261 material will follow ``Numerical methods with Matlab" by G. Recktenwald (Prentice Hall).
\item
More advanced material will follow ``Numerical Linear Algebra" by Trefethen and Bau (SIAM).
Any student who wants to buy this beautiful book, should ask me about SIAM membership.
\item
Other books that can be referred to are ``Matlab guide" by D. Higham and N. Higham;
Learning \LaTeX\ by Griffiths and D. Higham; ``Handbook of Writing for the Mathematical Sciences"
by N. Higham.
\end{itemize}

\section{Numerical topics}
For each topic, the course is split into a section at AM261 level, and one at AM561 level.
The chapters refer to Recktenwald.
\begin{itemize}
\item (Ch 5) Floating point numbers. Precision. (561) IEEE standard.
\item (Ch 5) Errors. (561) Forward and backward error. Condition number.
\item (561) Numerical differentiation and derivative formulae for differential equations.
\item (Ch 6) Root finding. Bisection, Newton. (561) Multiple Newton. Halley.
\item (Ch 7,8) Numerical Linear Algebra. Multiplication; Norms; Turing factors; Condition number.
(561) QR, SVD factors. Comparison of algorithms.
\item (Ch 10) Interpolation. polyfit, spline. Basis.
\item (Ch 9) Least-squares. Normal equations. (561) QR factors.
\item (Ch 11) Numerical integration. Newton-Cotes; Gaussian. (561) Romberg.
\item (Ch 12) Numerical solution of ODEs. Euler. (561) Stiff and non-stiff systems.
\item (561) Eigenvalues.
\end{itemize}

\section{Related topics}
These topics are an integral part of the course and are designed to equip the student
with essential research skills.
\begin{itemize}
\item Use of \LaTeX2e. All assignments must be submitted in \LaTeX.
\item Use of Matlab and other languages, including debugging tools.
\item Use of software products (Good and bad):  Netlib, Numerical Recipes.
\end{itemize}

\section{Assessment}
The assessment in this course will be based on the following.
\begin{itemize}
\item A midterm examination, mostly of AM261 material. The midterm takes place on
Saturday November 1. There is a written component 9--12 and a computer component 2 -- 5.
%\item A time-limited final examination, written unaided and `hands on' at a computer,
%in which you will
%demonstrate your ability to use computational software, and to write your conclusions in
%\LaTeX.
\item Four computational assignments, submitted in \LaTeX.
\end{itemize}


\end{document}
